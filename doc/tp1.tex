\documentclass[a4paper,10pt]{article}

\usepackage{graphicx}
\usepackage[ansinew]{inputenc}
\usepackage[spanish]{babel}
\usepackage{listings}

\title{		\textbf{Trabajo Pr\'{a}ctico 1: Conjunto de instrucciones MIPS}}

\author{	Hern\'{a}n Gonz\'{a}lez, \textit{Padr\'{o}n Nro. 79.460}		\\
            \texttt{ gonzalezhg@yahoo.com.ar }						\\[2.5ex]
            Pablo Magnaghi, \textit{Padr\'{o}n Nro. 88.126}				\\
            \texttt{ pablomagnaghi@gmail.com }						\\[2.5ex]
            Enzo Guagnini, \textit{Padr\'{o}n Nro. 88.325}				\\
            \texttt{ enzog\_m@hotmail.com }						\\[2.5ex]
            \normalsize{1er. Cuatrimestre de 2011}					\\
	    \normalsize{66.20 Organizaci\'{o}n de Computadoras  $-$ Pr\'{a}ctica Martes}\\
            \normalsize{Facultad de Ingenier\'{i}a, Universidad de Buenos Aires}	\\
       }

\date{03/05/2011}

\begin{document}

\maketitle

\thispagestyle{empty}

\newpage
\newpage

\section{Introducci\'{o}n}
Este art\'{i}culo se refiere a la implementaci\'{o}n parcial de un programa desarrollado en lenguaje assembly MIPS, el cual es una versi\'{o}n simplificada del 
comando \texttt{tr} de UNIX del trabajo anterior.
Para esto, se proveer\'{a} el c\'{o}digo en C de referencia de dicho comando, que consta fundamentalmente de los archivos \textit{main.c}, \textit{tr.c} y \textit{tr.h}.

\section{Enunciado}

\subsection{Objetivos}
Familiarizarse con el conjunto de instrucciones MIPS y el concepto de ABI, extendiendo un programa que resuelva el problema descripto en la secci\'{o}n 4.

\subsection{Alcance}
Este trabajo pr\'{a}ctico es de elaboraci\'{o}n grupal, evaluaci\'{o}n individual, y de car\'{a}cter obligatorio para todos alumnos del curso.

\subsection{Requisitos}
El informe deber\'{a} ser entregado personalmente, por escrito, en la fecha estipulada, con una  car\'{a}tula que contenga los datos completos de todos los 
integrantes. Adem\'{a}s, es necesario que el trabajo pr\'{a}ctico incluya (entre otras cosas), la presentaci\'{o}n de los resultados obtenidos, 
explicando, cuando corresponda, con fundamentos reales, las causas o razones de cada resultado obtenido.

\subsection{Descripci\'{o}n}
En este trabajo, se reimplementar\'{a} parcialmente en assembly MIPS la versi\'{o}n minimalista del comando tr de UNIX del 
trabajo anterior [1]. Para esto, se proveer\'{a} el cd\'{o}igo en C de referencia de dicho comando, que consta fundamentalmente 
de los archivos \textit{main.c}, \textit{tr.c} y \textit{tr.h}. 
Los alumnos deber\'{a}n reescribir en assembly MIPS32 las funciones definidas en \textit{tr.c}, respetando la interfaz definida 
en \textit{tr.h}, de forma tal que \textit{tr.c} ser\'{a} reemplazado por c\'{o}digo MIPS32 equivalente contenido en \textit{tr.S}, de 
manera transparente para \textit{main.c} (es decir, \textit{main.c} se mantendr\'{a} sin modificaciones, y se linkear\'{a} 
con \textit{tr.S} en lugar de \textit{tr.c}, para formar el ejecutable final).

\subsection{Implementaci\'{o}n}

El programa a implementar deber'{
a} satisfacer algunos requerimientos m'{i}nimos, que detallamos a continuaci'{o}n:

\subsubsection{ABI}
Ser\'{a} necesario que el c\'{o}digo presentado utilice la ABI explicada en la clase del martes 12/4/2011 [2].

\subsubsection{Casos de prueba}
Es necesario que la implementaci\'{o}n propuesta pase todos los casos incluidos tanto en el enunciado del trabajo anterior [1] como en el conjunto de pruebas
suministrado en el informe del trabajo, los cuales deber\'{a}n estar debidamente documentados y justificados.

\subsubsection{Documentaci\'{o}n}
El informe deber\'{a} incluir una descripci\'{o}n detallada de las t\'{e}cnicas y procesos
de desarrollo y debugging empleados, ya que forman parte de los objetivos principales del trabajo.

\subsection{Informe}
El informe deber\'{a} incluir:

\begin{itemize}
\item Este enunciado.
\item Documentaci\'{o}n relevante al dise\~no, desarrollo y debugging del programa.
\item Las corridas de prueba, (secci\'{o}n 5.2) y sus correspondientes comentarios.
\item El c\'{o}digo fuente completo, en dos formatos, digitalizado e impreso en papel.
\end{itemize}

\subsection{Fechas}
La fecha de entrega, es el martes 3/05. La fecha de vencimiento, 17/05.

\section{Documentaci\'{o}n}
La realizaci\'{o}n  de este informe se hizo con la herramienta TEX / LATEX. Tanto el arhivo en formato pdf como el archivo .tex se encuentran en la carpeta doc del cd entregado.

\section{T\'{e}cnicas y procesos de desarrollo y debugging empleados}

Para la compilaci\'{o}n del trabajo se fue trabajando con las funciones que contiene el \textit{tr.c} de manera separada reimplementandolas parcialmente en 
assembly para encontrar posibles errores, es decir, primero empezamos traduciendo a MIPS la funci\'{o}n \textit{void tr\_ds(char*, char*)} y se prob\'{o} que el programa 
funcionara en forma correcta pasando los casos de prueba pertinentes y luego se prosigui\'{o} con \textit{void tr\_d(char*)} siguiendo la misma 
metodolog\'{i}a hasta cubrir todas las funciones. 
Una vez que compilaban todas de manera correcta y pasaban los casos de prueba se junto todo el c\'{o}digo assembly en el archivo \textit{tr.S} y se realiz\'{o} un 
testeo general igual que en el tp0 y las respuestas fueron las esperadas.

Para la obtenci\'{o}n del ejecutable se ejecut\'{o} lo siguiente:

\textit{gcc -o tp1a main.c tr.S}

\section{Casos de prueba}
Los casos de prueba utilizados son los que figuran a continuaci\'{o}n.\\

\indent\texttt{\$ echo ``3.14159192'' | ./tp1a 1 9}\\
\indent\texttt{3.94959992}\\

\indent\texttt{\$ echo ``aba'' | ./tp1a aba 123}\\
\indent\texttt{323}\\

\indent\texttt{\$ echo ``aa bb cc'' | ./tp1a -s ab}\\
\indent\texttt{a b cc}\\

\indent\texttt{\$ echo ``aa bb cc'' | ./tp1a -s ab 12}\\
\indent\texttt{1 2 cc}\\

\indent\texttt{\$ echo ``Hola mundo'' | ./tp1a -d Ho}\\
\indent\texttt{la mund}\\

\indent\texttt{\$ echo ``aabbcc'' | ./tp1a -ds a b}\\
\indent\texttt{bcc}\\

\indent\texttt{\$ echo ``aaaa'' | ./tp1a -s a b}\\
\indent\texttt{b}\\

\indent\texttt{\$ echo ``bbb bbb'' | ./tp1a -s b}\\
\indent\texttt{b b}\\

\indent\texttt{\$echo ``bbbaaabb bbabbbbbbb'' | ./tp1a -ds b a}\\
\indent\texttt{a a}\\

\indent\texttt{\$echo ``f'' | ./tp1a aba def}\\
\indent\texttt{f}\\

\indent\texttt{\$echo ``abaco'' | ./tp1a -s a}\\
\indent\texttt{abaco}\\

\indent\texttt{\$echo ``abbbbbabbaaaco'' | ./tp1a -s a b}\\
\indent\texttt{bco}\\

\indent\texttt{\$echo ``bbbbbbaaaaaaabbb'' | ./tp1a -ds a b}\\
\indent\texttt{b}\

\indent\texttt{\$echo ``abaco baco abaco'' | ./tp1a a b}\\
\indent\texttt{bbbco bbco bbbco}\

\newpage
\newpage

\section{C\'{o}digo fuente}
A contiuaci\'{o}n se detalla el c\'{o}digo fuente implementado en lenguaje C que se encuentra en carpeta code del cd entregado 
que consta fundamentalmente de los archivos \textit{main.c}, \textit{tr.h} y \textit{tr.c}.

\lstset{basicstyle=\scriptsize, commentstyle=\textit, numbers=left, numberstyle=\footnotesize, tabsize=4, breaklines=true}
\lstset{language=C} 
\lstset{extendedchars=false} 
\lstinputlisting {../code/main.c}
\lstinputlisting {../code/tr.h}
\lstinputlisting {../code/tr.c}

\newpage
\newpage

\subsection{C\'{o}digo MIPS}
El siguiente es el c\'{o}digo assembly que reimplementa parcialmente en assembly MIPS la versi\'{o}n minimalista del comando tr,
este archivo se encuentra en la carpeta code del cd entregado:
\lstset{basicstyle=\scriptsize, commentstyle=\textit, numbers=left, numberstyle=\footnotesize, tabsize=4, breaklines=true}
\lstset{language=C} 
\lstset{extendedchars=false} 
\lstinputlisting {../code/tr.S}

\newpage
\newpage

\section{Conclusiones}
El presente logr\'{o} familiarizarnos a distintas herramientas que se utilizan en la materia.
Entre las actividades realizadas podemos resaltar:

\begin{itemize}
\item Realizar c\'{o}digo en assembly para la arquitectura MIPS.
\item Aprender a utilizar la ABI.
\item Desarrollar el presente informe en LATEX.
\end{itemize}

\end{document}
